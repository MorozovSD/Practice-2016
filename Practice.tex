\documentclass{article}
%добавляет русский язык
\usepackage[english,russian]{babel}
\usepackage[14pt]{extsizes}
\usepackage{color}
\definecolor{darkgreen}{rgb}{0,.5,0}
\usepackage[colorlinks,linkcolor=black,urlcolor=blue]{hyperref}
%%TODO Настроить отступы 
%%Добавление абзацев после оглавления (indentfirst)
\usepackage[indentfirst]{titlesec}
\usepackage{titletoc}
%%TODO Настроить нумерацию
\usepackage[utf8]{inputenc}
\linespread{1.3}
\begin{document}
\begin{titlepage}
	\centering
	{\LARGE Университет ИТМО \par}
	\vspace{5mm}
	{\Large Кафедра вычислительной техники\par}
	\vspace{1.5cm}
	{\huge\bfseries Отчет по прохождению практики\par}
	\vspace{3cm}
	\begin{flushleft}
		\hangindent=10cm
		\hangafter=-5
		\noindent 
		{\Large Студента\\
				P3311 группы \\
				Морозова С.Д.\\
				Руководитель \\
				Соснин В.В.
		}
	\end{flushleft}
	\vfill
% Bottom of the page
	\vspace{1cm}
	{\large Санкт-Петербург \par}
	{\large 2016  \par}
\end{titlepage}
%%Нужны ли в содержании subsubsection'ы?
	\setcounter{tocdepth}{3}	
	\tableofcontents
	\newpage
	\section{Введение}
	\indent 
	%%Тема паралельные вычисления. Перед тем как приступить к тебе необходимо разобраться с латех и гит...
		Тема прохождения практики "--- параллельные вычисления. Цель задания "--- сравнить различные функции в языке С, которые 		можно использовать для измерения времени работы параллельных программ.
		
		Однако требования руководителя практики таковы, что перед тем как приступить к выполнению основного задания нужно 				ознакомиться с системой компьютерной вёрстки TeX (LaTeX), которая должна
	использоваться для написания отчёта, и ознакомиться с системой контроля версий Git, с последующим созданием учетной записи на 	сайте GitHub или анагичном.
	\newpage
	\section{Система компьютерной верстки \TeX(\LaTeX)}	
		\subsection{Краткое описание}
			\TeX ~"--- система компьютерной вёрстки с формулами, разработанная американским профессором информатики Дональдом 				Кнутом. Название происходит от греческого слова $\tau\varepsilon\chi\upsilon\eta$ "--- "<искусство">, "<мастерство">, 				поэтому	последняя буква читается как русская Х. Хотя TeX является системой набора и верстки, развитые возможности 					макроязыка TeX делают его Тьюринг-полным языком программирования. 
		
			\TeX ~работает с боксами (box) и клеем (glue). Бокс "--- двумерный объект прямоугольной формы, характеризуется тремя 
		величинами (высота, ширина, глубина). Элементарные боксы "--- это буквы, которые объединяются в боксы-слова, которые в 				свою очередь сливаются в боксы-строчки, боксы-абзацы и т.д.

        	Между боксами располагается клей, который имеет некоторую ширину по умолчанию и степени увеличения/уменьшения этой 				ширины. Объединяясь в бокс более высокого порядка, боксы могут шевелиться, но после того как найдено оптимальное решение, 		это состояние закрепляется, и полученный бокс выступает как единое целое.
        
       		Инетересный факт. На версии 3.0 дизайн был заморожен, поэтому в новых версиях не будет добавления новой 						функциональности, только исправление ошибок. Версия \TeX 'a ассимтотически приближается к числу $\pi$. Это факт говорит о 		том, что последняя версия	3.14159265 (январь 2014) является крайне стабильной и возможны лишь мелькие исправления. 				Дональд Кнут заявил, что последнее обновление (сделанное после его смерти) сменит номер версии на ~$\pi$, и с этого 				момента все ошибки станут особенностями.
        		
			\LaTeX ~"--- созданный Лесли Лэмпортом набор макрорасширений (или макропакет) системы компьютерной вёрстки \TeX, 				который облегчает набор сложных документов. Стоит отметить, что как и любой другой макропакет\footnote{ Так же существуют 		Plain TeX, AMS-TeX, AMS-LaTeX и т.д.} \LaTeX ~не может расширить возможности \TeX ~(все, что можно сделать в одном пакете 		можно сделать и в любом другом). Пакет позволяет автоматизировать многие задачи набора текста и подготовки статей, 					включая набор текста на нескольких языках, нумерацию разделов и формул, размещение иллюстраций и таблиц на странице, 				ведение библиографии и др. Все это делает \LaTeX ~крайне удобным инструментом для написания научных статей, диссертаций и 		т.п..
		\newpage
			
		\subsection{Сравнение \LaTeX ~и MS Word}
			В качестве сравнения "--- перечислим плюсы и минусы \LaTeX ~перед MS Word(а так же всеми его аналогами). \\	
	    Плюсы \LaTeX: 
	    \begin{itemize} 
	    	\item	Проста работы с любыми математическими формулами
	    	\item	Кроссплатформенность 
	    	\item	Без особых трудностей можно получить сноски, список литературы,
					оглавление, список таблиц, указатель и т. п.
	    	\item	Имеется несколько стандартных стилей (книга, статья, доклад,
					письмо), с помощью которых получаются документы очень высокого
					полиграфического качества 
	    	\item	Гибкая работа с логикической структурой текста
	    	\item	Язык международного обмена по математике и физике (большинство     
   					научных издательств принимают тексты в печать  только в этом формате)
    	\end{itemize}
    	\newpage
    	Минусы \LaTeX:
		\begin{itemize} 
	    	\item	Не является системой типа WYSIWYG
	    				\footnote{What You See Is What You Get(Что видишь, то и получишь). Стоит отметить, что существуют 									дистрибутивы \TeX ~в которых есть попытки реализовать WYSIWYG. Например платный дистрибутив  BaKoMa TeX + 						текстовый редактор  BaKoMa TeX Word.}   
	    	\item	При серьезных отклонениях от стандартных стилей документов требуется
					достаточно сложное программирование	
    	\end{itemize}
    	
    		То есть, выбирая между \LaTeX ~и MS Word, стоит обратить внимание на то,какой текст вы собираетесь печатать, 					насколько нестандартный будет стиль текста, на его примерный объем. В некоторый случаях достаточно использовать MS Word,   		в других "--- использование \LaTeX ~может заметно упростить работу.
		\subsection{Выбор инструмента редактирования}
			В ходе изучения всех возможных вариантов работа с \LaTeX для создния данного отчета, была выбрана программа Textmaker				\footnote{Оффициальный сай Textmaker:~ \href{http://www.xm1math.net/texmaker/}{http://www.xm1math.net/texmaker/}}.\\
		Выбор Textmaker'а обусловлен следующими его особенностями:
			\begin{itemize} 
	    		\item	Автоматическая подсветка синтаксиса
	    		\item	Функция автодополнения команд \LaTeX
	    		\item	Сокрытие блоков кода (Code folding)
	    		\item	Быстрая навигация по структуре документа
	    		\item	Указание на строку с ошибкой, для быстрой отладки
	    		\item	Интегрированный просмотр PDF
			\end{itemize} 
	\newpage
	\section{Системы контроля версий}
		\subsection{Краткое описание}
		\subsection{Достоинста и недостатки Git}
		\subsection{GitHub}
	\section{Паралельные вычисления}
		\subsection{История}
		\subsection{Что-нибудь из теории}
		\subsection{Что-нибудь еще...}
	\newpage
	\section{Функции замера времени}
		%%Описывание разных функций замера времени
		\subsection{Принцип работы}
		%%Работают только в Windows		
		\subsection{Windows}
			\subsubsection{func1}
			\subsubsection{func2}
			\subsubsection{...}
		%%Работают только в Linux
		\subsection{Linux}
			\subsubsection{func4}
			\subsubsection{func5}
			\subsubsection{...}
		%%Работают и в Windows и в Linux
		\subsection{Кросплатформенные}
			\subsubsection{func7}
			\subsubsection{func8}
			\subsubsection{...}
		\subsection{Проблемы и сложности замеров времени \\ при параллельный вычислениях}
	\newpage
	\section{Практическая часть?}
		\subsection{Описание эксперементальной программы}
		\subsection{Результаты работы программы}
		%%Большая таблица сравнения развых функций замеров времени
		\subsection{Выводы}
		%%Сравнение функций
	\newpage
	\section{Вывод по производственной практике}
	%%Вывод по всей работе
	\newpage
	\section{Список литературы}
	%%Ссылки, чтобы ничего не забыть
	%%http://mgena.chat.ru/latex/indru.html
	%%http://zns.susu.ru/IT/latex/literature/PartlS_LaTeX.pdf
	%%https://ru.sharelatex.com/learn/Page_size_and_margins
	%%http://elibrary.bsu.az/kitablar/1021.PDF
	%%http://zns.susu.ru/IT/latex/Marking/marking.html
	%%https://en.wikibooks.org/wiki/LaTeX/Title_Creation
	%%http://www.sbras.ru/win/docs/TeX/LaTex2e/Text_in_LaTeX.pdf
	%%TODO сделать красиво и по ГОСТ'у
\end{document}

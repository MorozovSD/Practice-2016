\documentclass{article}
\usepackage[english,russian]{babel}
\usepackage[14pt]{extsizes}
\usepackage{geometry}
%%TODO Настроить отступы и интервалы
%%TODO Настроить нумерацию
\usepackage[utf8]{inputenc}
\begin{document}
\begin{titlepage}
	%%Нужны ли в содержании subsubsection'ы?
	\centering
	{\LARGE Университет ИТМО \par}
	\vspace{1cm}
	{\Large Кафедра вычислительной техники\par}
	\vspace{1.5cm}
	{\huge\bfseries Отчет по прохождению практики\par}
	\vspace{3cm}
	\begin{flushleft}
		\hangindent=10cm
		\hangafter=-5
		\noindent 
		{\Large Студента\\
				P3311 группы \\
				Морозова С.Д.\\
				\vspace{1cm}
				Руководитель \\
				Соснин В.В.
		}
	\end{flushleft}
	\vfill
% Bottom of the page
	\vspace{2.5cm}
	{\large Санкт-Петербург \par}
	{\large 2016  \par}
\end{titlepage}
	\setcounter{tocdepth}{3}
	\tableofcontents
	\newpage
	\section{Введение}	
	%%Тема паралельные вычисления. Перед тем как приступить к тебе необходимо разобраться с латех и гит...
	\section{LaTeX}
		\subsection{Краткое описание}
		\subsection{Сравнение LaTex и MS Word}
		\subsection{Выбор инструмента редактирования}
	\newpage
	\section{Системы контроля версий}
		\subsection{Краткое описание}
		\subsection{Достоинста и недостатки Git}
		\subsection{GitHub}
	\newpage
	\section{Паралельные вычисления}
		\subsection{История}
		\subsection{Что-нибудь из теории}
		\subsection{Что-нибудь еще...}
	\newpage
	\section{Функции замера времени}
		%%Описывание разных функций замера времени
		\subsection{Принцип работы}
		%%Работают только в Windows		
		\subsection{Windows}
			\subsubsection{func1}
			\subsubsection{func2}
			\subsubsection{...}
		%%Работают только в Linux
		\subsection{Linux}
			\subsubsection{func4}
			\subsubsection{func5}
			\subsubsection{...}
		%%Работают и в Windows и в Linux
		\subsection{Кросплатформенные}
			\subsubsection{func7}
			\subsubsection{func8}
			\subsubsection{...}
		\subsection{Проблемы и сложности замеров времени \\ при параллельный вычислениях}
	\newpage
	\section{Практическая часть?}
		\subsection{Описание эксперементальной программы}
		\subsection{Результаты работы программы}
		%%Большая таблица сравнения развых функций замеров времени
		\subsection{Выводы}
		%%Сравнение функций
	\newpage
	\section{Вывод по производственной практике}
	%%Вывод по всей работе
	\newpage
	\section{Список литературы}
	%%Ссылки, чтобы ничего не забыть
	%%http://mgena.chat.ru/latex/indru.html
	%%http://zns.susu.ru/IT/latex/literature/PartlS_LaTeX.pdf
	%%https://ru.sharelatex.com/learn/Page_size_and_margins
	%%http://elibrary.bsu.az/kitablar/1021.PDF
	%%http://zns.susu.ru/IT/latex/Marking/marking.html
	%%https://en.wikibooks.org/wiki/LaTeX/Title_Creation
	%%http://www.sbras.ru/win/docs/TeX/LaTex2e/Text_in_LaTeX.pdf
	%%TODO сделать красиво и по ГОСТ'у
\end{document}
